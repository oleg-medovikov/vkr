% НАЧАЛО ТИТУЛЬНОГО ЛИСТА
\begin{center}
\hfill \break
\small{\textbf{Санкт-Петербургский государственный электротехнический университет}}\\
\small{\textbf{"ЛЭТИ" им. В. И. Ульянова (Ленина)}}\\
\small{\textbf{(СПБГЭТУ "ЛЭТИ")}}\\
\hfill \break

\begin{center}
\begin{tabular}{lr}
Направление: & \textbf{27.04.04} - Управление в технических системах \\
Профиль: &  Управление и информационные технологии в технических системах \\
Факультет: & Компьютерных технологий и информатики \\
Кафедра: & Автоматики и процессов управления \\
\\
К защите допустить & \\
Зав. кафедрой &  Шестопалов М. Ю.
\end{tabular}
\end{center}


\normalsize{}\\
 \hfill \break
\hfill\break
\hfill \break
\hfill \break
\hfill \break
\large{\textbf{ВЫПУСКНАЯ КВАЛИФИКАЦИОННАЯ РАБОТА МАГИСТРА}}\\
\hfill \break
\hfill \break
\normalsize{\textbf{Тема: Параметрическое проектирование дельта-робота и решение задачи координатного управления рабочим органом}}\\
\hfill \break
\hfill \break
\hfill \break
\hfill \break
 
\normalsize{ 
	\begin{tabular}{p{150pt} p{100}ll}
Студент & & \underline{\hspace{3cm}} & О.Е. Медовиков\\\\
Руководитель & к. т. н. & \underline{\hspace{3cm}} &С. Е. Абрамкин\\\\
\end{tabular}
}\\
\hfill \break
\hfill \break
\hfill \break
\hfill \break
\hfill \break
\hfill \break
Санкт-Петербург \\ 2020 
\end{center}
\thispagestyle{empty} % выключаем отображение номера для этой страницы
 
% КОНЕЦ ТИТУЛЬНОГО ЛИСТА

\newpage
% НАЧАЛО ЗАДАНИЯ
\begin{center}
\large{\textbf{ЗАДАНИЕ НА ВЫПУСКНУЮ КВАЛИФИКАЦИОННУЮ РАБОТУ}}\\
\end{center}
\begin{flushright}
Утверждаю\\
Заф. кафедры АПУ\\
\underline{\hspace{3cm}} Шестопалов М. Ю.\\
«\underline{\hspace{0.7cm}}»\underline{\hspace{3cm}}2020 г.
\end{flushright}

\hfill \break
Студент Медовиков О. Е. \hspace{7cm} Группа 4391\\
Тема работы:\\ Параметрическое проектирование дельта-робота и решение задачи координатного управление рабочим органом.\\
Исходные данные (технические требования): \\
1. Написание программы для параметрического моделирования дельта-робота\\
2. Написание программы для управления дельта-роботом\\
3. Создание рабочей модели дельта-робота\\
Содержание ВКР:\\
\\\\\\
Перечень отчетных материалов: пояснительная записка, иллюстративный материал, приложение.\\
Дополнительные разделы:\\\\

\begin{tabular}{p{240pt}l}
	Дата выдачи задания & Дата предоставления ВКР к защите\\
	«\underline{\hspace{0.7cm}}»\underline{\hspace{3cm}}2020 г. &  «\underline{\hspace{0.7cm}}»\underline{\hspace{3cm}}2020 г.\\
\end{tabular}
\hfill \break
\hfill \break
\hfill \break
\hfill \break
\begin{tabular}{p{150pt} p{100} ll}
Студент & & \underline{\hspace{3cm}} & О.Е. Медовиков\\\\
Руководитель & к. т. н. & \underline{\hspace{3cm}} &С. Е. Абрамкин\\\\
\end{tabular}
\thispagestyle{empty} % выключаем отображение номера для этой страницы

\newpage
\begin{center}
\large{\textbf{КАЛЕНДАРНЫЙ ПЛАН ВЫПОЛНЕНИЯ ВЫПУСКНОЙ КВАЛИФИКАЦИОННОЙ РАбОТЫ}}\\
\end{center}
\begin{flushright}
Утверждаю\\
Заф. кафедры АПУ\\
\underline{\hspace{3cm}} Шестопалов М. Ю.\\
«\underline{\hspace{0.7cm}}»\underline{\hspace{3cm}}2020 г.
\end{flushright}

\hfill \break
Студент Медовиков О. Е. \hspace{7cm} Группа 4391\\
Тема работы:\\ Параметрическое проектирование дельта-робота и решение задачи координатного управление рабочим органом.\\

\hfill \break
\begin{tabular}{|p{20pt} |p{300}| l|}
	\hline
	№ п/п & Наименование работ & Срок выполнения\\ \hline
	1 & Обзор литературы по теме работы & 10.12-01.02\\ \hline
	2 & &\\ \hline
	3 & &\\ \hline
	4 & &\\ \hline
\end{tabular}
\hfill \break
\hfill \break
\hfill \break
\hfill \break
\hfill \break
\hfill \break
\begin{tabular}{p{150pt} p{100} ll}
Студент & & \underline{\hspace{3cm}} & О.Е. Медовиков\\\\
Руководитель & к. т. н. & \underline{\hspace{3cm}} &С. Е. Абрамкин\\\\
\end{tabular}
\thispagestyle{empty} % выключаем отображение номера для этой страницы

