% НАЧАЛО ТИТУЛЬНОГО ЛИСТА
\begin{center}
\hfill \break
\small{\textbf{Санкт-Петербургский государственный электротехнический университет}}\\
\small{\textbf{"ЛЭТИ" им. В. И. Ульянова (Ленина)}}\\
\small{\textbf{(СПБГЭТУ "ЛЭТИ")}}\\
\hfill \break

\begin{center}
\begin{tabular}{lr}
Направление: & \textbf{27.04.04} - Управление в технических системах \\
Профиль: &  Управление и информационные технологии в технических системах \\
Факультет: & Компьютерных технологий и информатики \\
Кафедра: & Автоматики и процессов управления \\
\\
К защите допустить & \\
Зав. кафедрой &  Шестопалов М. Ю.
\end{tabular}
\end{center}

\normalsize{}
\vspace{3cm}
\large{\textbf{ВЫПУСКНАЯ КВАЛИФИКАЦИОННАЯ РАБОТА МАГИСТРА}}\\
\vspace{1cm}
\normalsize{\textbf{Тема: Параметрическое проектирование дельта-робота и решение задачи координатного управления рабочим органом}}\\
\vspace{3cm}
 
\begin{flushleft}
 \hspace{1cm} Студент \hspace{7cm} \underline{\hspace{3cm}}  О.Е. Медовиков \\ 
 \vspace{5mm}
 \hspace{1cm} Руководитель \hspace{2cm} к. т. н. \hspace{2cm} \underline{\hspace{3cm}}  С. Е. Абрамкин\\ 
 \vspace{5mm}
 \hspace{1cm} Консультант \hspace{2.3cm} к. э. н. \hspace{2cm} \underline{\hspace{3cm}}  Ю. Р. Ичкитидзе\\ 
\end{flushleft}

\vspace{2cm}
Санкт-Петербург \\ 2020 
\end{center}
\thispagestyle{empty} % выключаем отображение номера для этой страницы
 
% КОНЕЦ ТИТУЛЬНОГО ЛИСТА

\newpage
% НАЧАЛО ЗАДАНИЯ
\begin{center}
\large{\textbf{ЗАДАНИЕ НА ВЫПУСКНУЮ КВАЛИФИКАЦИОННУЮ РАБОТУ}}\\
\end{center}
\begin{flushright}
Утверждаю\\
Заф. кафедры АПУ\\
\underline{\hspace{3cm}} Шестопалов М. Ю.\\
«\underline{\hspace{0.7cm}}»\underline{\hspace{3cm}}2020 г.
\end{flushright}

\hfill \break
Студент Медовиков О. Е. \hspace{7cm} Группа 4391\\
Тема работы:\\ Параметрическое проектирование дельта-робота и решение задачи \hfill \break координатного управление рабочим органом.\\
Исходные данные (технические требования): \\
1. Написание программы для параметрического моделирования дельта-робота\\
2. Написание программы для управления дельта-роботом\\
3. Создание рабочей модели дельта-робота\\
Содержание ВКР:\\
\vspace{2cm} \\
Перечень отчетных материалов: пояснительная записка, иллюстративный\\ материал, приложение.\\
Дополнительные разделы:\\
\vspace{2cm}

\begin{tabular}{p{230pt}c}
Дата выдачи задания & Дата предоставления ВКР к защите\\
«\underline{\hspace{0.7cm}}»\underline{\hspace{3cm}}2020 г. &  «\underline{\hspace{0.7cm}}»\underline{\hspace{3cm}}2020 г.\\
\end{tabular}
\vspace{1cm}

\begin{flushleft}
 \hspace{1cm} Студент \hspace{7cm} \underline{\hspace{3cm}}  О.Е. Медовиков \\ 
 \vspace{5mm}
 \hspace{1cm} Руководитель \hspace{2cm} к. т. н. \hspace{2cm} \underline{\hspace{3cm}}  С. Е. Абрамкин\\ 
 \vspace{5mm}
 \hspace{1cm} Консультант \hspace{2.3cm} к. э. н. \hspace{2cm} \underline{\hspace{3cm}}  Ю. Р. Ичкитидзе\\
\end{flushleft}

\thispagestyle{empty} % выключаем отображение номера для этой страницы

\newpage
\begin{center}
\large{\textbf{КАЛЕНДАРНЫЙ ПЛАН ВЫПОЛНЕНИЯ ВЫПУСКНОЙ КВАЛИФИКАЦИОННОЙ РАбОТЫ}}\\
\end{center}
\begin{flushright}
Утверждаю\\
Заф. кафедры АПУ\\
\underline{\hspace{3cm}} Шестопалов М. Ю.\\
«\underline{\hspace{0.7cm}}»\underline{\hspace{3cm}}2020 г.
\vspace{1cm}
\end{flushright}

\begin{flushleft}
Студент Медовиков О. Е. \hspace{7cm} Группа 4391\\
Тема работы:\\ Параметрическое проектирование дельта-робота и решение задачи\\ координатного управление рабочим органом.\\
\vspace{1cm}
\end{flushleft}


\begin{tabular}{| c | l | c | }
\hline
№ п/п & Наименование работ & Срок выполнения\\
\hline
1 & Обзор литературы по теме работы & 10.12 - 01.02\\ 
\hline
2 & Проектирование виртуальной модели& 10.12 - 26.03\\
\hline
3 & Создание физического прототипа робота & 01.02 - 05.04\\
\hline
4 & Написание прошивки для микроконтроллера & 25.04 - 15.05 \\
\hline
5 & Создание интерфейса для управления роботом &15.05 - 25.05 \\
\hline
\end{tabular}

\vspace{3cm}

\begin{flushleft}
 \hspace{1cm} Студент \hspace{7cm} \underline{\hspace{3cm}}  О.Е. Медовиков \\ 
 \vspace{5mm}
 \hspace{1cm} Руководитель \hspace{2cm} к. т. н. \hspace{2cm} \underline{\hspace{3cm}}  С. Е. Абрамкин\\ 
 \vspace{5mm}
 \hspace{1cm} Консультант \hspace{2.3cm} к. э. н. \hspace{2cm} \underline{\hspace{3cm}}  Ю. Р. Ичкитидзе\\
\end{flushleft}

\thispagestyle{empty} % выключаем отображение номера для этой страницы

