\addcontentsline{toc}{section}{Заключение}
\begin{center}
\large{\textbf{ЗАКЛЮЧЕНИЕ}}\\
\end{center}

После изготовления модели дельта-робота были выявлена недостаточная жесткость конструкции. <<Лепестки>>, на которых расположены двигатели, имеют явный люфт в месте сочлинения с базой, который обусловлен неправильным позиционированием пазов. На данный момент паз сделан так, что эффективно препядствует движению в горизонтальной плоскости, но при этом дает вертикальные колебания. <<Ласточкин хвост>> требуется перевернуть на $90^{\circ}$, чтобы это исправить.
Изучив движения дельта-робота, я пришел к выводу, что величина радиуса базы должна быть значительно больше. Не в пределах от 90 мм. до 150 мм., а начинаться от 150 мм. и доходить до 300 мм. При таких плечах, делать детали целиком из пластика не имеет смысла. В новом видении дизайна робота, база должна иметь металлический скелет, а пластик будет играть роль сухожилий, связывающих все прочие детали. На стоимость робота это не должно повлиять, но упростит и ускорит процесс печати.

Написанный  скетч реализует минимальный функционал, необходимый для управления роботом. Arduino умеет возвращать каретку в начальное положение, отсчитывать координаты, и перемещаться на заданные через com-порт величины углов. Важно, что благодаря реализованной фунции, двигатели совершают шаги единовременно, без простоя. Есть пространство для улучшения механизма разгона двигателей, возможно, получится реализовать дополнительно торможение по квадратичному закону. Ощущается упор в возможности драйвера a4988, возможно стоит расмотреть другие варианты драйверов.  

