\section{Моделирование робота}
\subsection{ZenCad}
В качестве CAD программы для моделирования деталей робота и создания цифрового двойника была выбрана библиотека параметрического 3d моделирования ZenCad. Автор библиотеки вдохновлен программой OpenScad, проектирование в котором заключалось в написании скрипта, являющегося инструкцией для графического ядра, строящего модель. Без интерактивных инструментов изменения объектов. В ZenCad используется ядро граничного представления OpenCascade, что является преимуществом по сравнению с OpenScad, использующим математику полигональных сеток. Необходимость представлять каждую модель, как массив полигонов приводит к комбинаторному взрыву, при усложнении сцены, что в свою очередь заставляет разрабатывать модели с меньшим разрешением, чем их финальный вид, ради экономии вычислительных ресурсов. Особенностью граничного представления твёрдого тела заключается в описании модели, как набора поверхностей, с заданной точностью соединенных по границам и образующих замкнутый объем. То-есть окружность задается не как многоугольник с заданным количеством вершин, а именно как функция окружности, задаваемая двумя точками: центром и точкой на окружности, соответствующей 0 радиан. Подобное представление позволяет определять объем тела и его массово-инерционные характеристики, а также упрощает его разбиение на конечные элементы для инженерного анализа (так как можно легко определить, лежит ли точка внутри тела или за его пределами). Для удобства работы параметрические модели представляются в виде логического дерева построения. Фактически, каждая геометрическая операция (вытянуть, построить по сечениям и др.) – это набор алгоритмов, создающих набор корректно связанных поверхностей. При изменении компонента дерева построения все последующие элементы будут перестроены. Если при этом исчезнут элементы, к которым были привязаны последующие построения, то модель окажется некорректной. Поэтому, проектируя изделие, необходимо четко представлять иерархию дерева и возможные способы последующего изменения геометрии.

Для ускорения расчетов сцены используется библиотека ленивых вычислений evalcache для агрессивного кэширования вычислений. Объекты, параметры которых не были изменены будут исключены из расчетов и загружены из кэша. В том числе и одинаковые объекты сцены не будут обсчитываться несколько раз. Ради демонстрации моделей с разными значениями параметров, можно заранее их обсчитать и соответственно разные версии модели будут храниться в кэше. В теории возможно переносить папку кэша с одного компьютера на другой, если это целесообразно (очень сложный проект и маломощный компьютер). С другой стороны, моделирование в ZenCad представляет собой программирование, то-есть изменение текстового документа, для чего возможно использовать любой компьютер и текстовый редактор.

Созданный мной цифровой двойник представляет собой программу на языке <<Python>>, текст которой разделён на три логических секций:

\paragraph{Константы} здесь размещены все константы, используемые в скрипте и их часто используемые отношения, для облегчения расчетов и экономии места.
\paragraph{Функции} для удобства работы с деталями робота, все они создаются в виде функций, к которым происходит обращение в нужный момент. К некоторым функциям, бывают обращения внутри других функций, например, для выреза под гайку, которая используется в нескольких деталях.  
\paragraph{Интерактивные объекты и анимация} особенностью ZenCad является то, что сами по себе модели не будут присутствовать в сцене до тех пор, пока не будут превращены в интерактивные объекты. Есть несколько способов создания интерактивных объектов, я использую функцию n = disp(m), которая из объекта, хранящейся в переменной m, создает интерактивный объект n. В случае с анимацией, если координату куба представить в виде массива из 10 значений, то в сцене будет генерироваться 10 кубов. Для создания анимации, необходимо менять координату именно интерактивного тела, что делается с помощью специальных функций. 

\subsection{База}



